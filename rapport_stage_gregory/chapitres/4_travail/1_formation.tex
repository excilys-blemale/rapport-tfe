\section{La formation}

\subsection{Contenu}

La formation dispensée a pour but principal de mettre tous les stagiaires au même niveau sur les technologies Java/JEE utilisées dans le monde de l'entreprise pour pouvoir nous recruter et facilement nous placer en clientèle en garantissant l'excellence technique chère au \excilysGroup{}.\\

Nous avons ainsi abordé les sujets suivants :

\begin{itemize}
	\item eXtreme Programming
	\item UML
	\item Subversion
	\item Maven 2
	\item Java 5
	\item Log4J
	\item JDBC
	\item JUnit 3 et 4
	\item JEE
	\item Hibernate 3
	\item Spring 2.5\\
\end{itemize}

Les technologies abordées sont pour la plupart survolées et certaines sont dépassées, bien que toujours utilisées. La formation sert de socle pour une auto-formation plus en détail sur ces même technologies. Cette auto-formation est une composante importante du métier de consultant puisqu'elle permet de rester au niveau, de se former sur de nouvelles technologies et, éventuellement, de proposer des alternatives sur la direction d'un projet.

\subsection{Déroulement}

L'ensemble de la formation se fait via la plateforme d'e-learning développée par \excilys{}, à savoir Capico \cite{capico}. Elle consiste en une suite de cours magistraux éventuellement sonorisés pour une meilleure compréhension. Ces cours sont accompagnés de QCM pour vérifier que les connaissances ont bien été assimilées ainsi que d'exercices pour la mise en pratique.

\begin{figure}[H]
	\centering
	\includegraphics[width=\linewidth]{images/capico.pdf}
	\caption{Éxemple de cours sur Capico}
\end{figure}

Stéphane LANDELLE est responsable de la formation des stagiaires. Il s'assure donc du bon suivi de la formation et la ponctue d'interventions visant à vérifier nos connaissances ou à éclaircir certains points qui pourraient être compliqués. Il permet aussi d'apporter un point de vue plus pratique sur ces technologies, notamment comment elles sont utilisées dans le monde de l'entreprise et comment elles devraient l'être.

\subsection{Technologies abordées}

Les technologies abordées lors de la formation étant utilisées pendant le reste du stage, je vais maintenant les présenter brièvement pour une meilleure compréhension de la suite du rapport.

\subsubsection{eXtreme Programming (XP)}

XP est une méthode Agile utilisée dans le développement logiciel assez proche d'une autre méthode Agile plus connue : la méthode Scrum \cite{scrum}. Elle découpe le développement d'une application en plusieurs sprints d'une à deux semaines. Le développement est ainsi incrémental garantissant la sortie d'une application fonctionnelle à chaque fin de sprint. Le développement incrémental permet d'avoir un développement plus souple.\\

XP s'appuie notamment sur le pair programming qui consiste à travailler en binôme sur un même ordinateur avec un \emph{pilote} qui contrôle le clavier et la souris et un \emph{copilote} qui a plus de recul et aide le \emph{pilote}. Régulièrement, les rôles s'inversent.\\

Un autre principe de cette méthode est le Test Driven Development (TDD). TDD vise à rendre le développement plus simple et efficace en écrivant d'abord les tests permettant de garantir qu'une fonctionnalité a été développée puis de développer la fonctionnalité. L'idée est de simplement écrire le code qui garantie que les tests passent.

\subsubsection{Maven}

Maven \cite{maven} est un outil de gestion et de construction de projets Java. Une de ces forces est de notamment gérer les librairies externes ou encore la sortie d'une nouvelle version. Maven permet d'automatiser la plupart des tâches liées à un projet informatique : récupération des librairies externes, compilation, exécution des tests, génération des livrables, déploiement de l'application sur un serveur, \dots C'est un outil quasiment indispensable pour un projet Java.

\subsubsection{Java}

Java \cite{java} est un langage de programmation orienté objet développé par Sun Microsystems (racheté par Oracle Corporation). Il est l'un des langage les plus utilisé en entreprise. C'est un langage compilé qui s'exécute sur une machine virtuelle appelée Java Virtual Machine (JVM), ce qui permet aux applications Java d'être portables puisque seule la JVM dépend du système d'exploitation.\\

Un des principe fondamental de Java qui fait de lui un langage très apprécié des entreprise est la garantie de la rétro-compatibilité entre différentes versions de Java. Ainsi une application développée en Java 1.4 peut fonctionner sur une JVM Java 7, profitant ainsi des améliorations apportées à celle-ci.

\subsubsection{JUnit}

JUnit \cite{junit} est un framework permettant d'effectuer facilement des tests unitaires. C'est actuellement le framework le plus utilisé pour les tests dans le monde Java ce qui lui permet d'être très bien intégré dans les environnements de développement tels qu'Eclipse, ou dans d'autres outils comme Maven.

\subsubsection{Java Enterprise Edition (JEE)}

JEE \cite{jee} est un ensemble de spécifications (28 spécifications pour JEE 6) appelées JSR permettant de normaliser des technologies utilisées dans les applications telles que :

\begin{description}
	\item[Servlets] objets Java capable de répondre à des requêtes HTTP,
	\item[Java Server Pages (JSP)] langage permettant de générer dynamiquement des pages HTML en s'interfaçant facilement avec du code Java,
	\item[Java Persistence API (JPA)] API inspirée d'Hibernate, permettant de persister dans une base de données relationnelle des objets Java,
	\item[Entreprise JavaBean (EJB)] architecture servant à créer des composants distribuées.
\end{description}

\subsubsection{Java DataBase Connectivity (JDBC)}

JDBC \cite{jdbc} est une API permettant d'écrire des requêtes vers des bases de données relationnelles. JDBC permet de s'abstraire du langage de requêtes utilisé par la base de données. La traduction des requêtes se fait en temps réel par le pilote utilisé pour se connecter à la base de données.

\subsubsection{Hibernate}

Hibernate \cite{hibernate} est un Object Relational Mapping (ORM) framework, c'est à dire un framework permettant de persister de façon simple et intuitive les objets Java dans une base de données relationnelle. Hibernate s'occupe de faire le mapping entre les classes et attributs Java et les tables d'une base de données. Hibernate est donc une surcouche à JDBC. Depuis sa version 4, Hibernate est une des implémentation de JPA.

\subsubsection{Spring}

Le framework Spring \cite{sprint} est un concurrent à JEE. L'idée est de proposer au développeur un ensemble de fonctionnalités à la manière de JEE mais sous la forme de dépendances modulaires. Ainsi, une application Spring peut fonctionner sur une simple JVM ou sur un serveur léger tel que Tomcat ou Jetty. Au contraire, une application JEE se déploie dans un conteneur lourd tel que Glassfish ou JBoss implémentant toutes les JSR d'un serveur JEE. Les applications JEE sont extrêmement dépendantes du conteneur ce qui enlève la portabilité de l'application, contrairement à une application Spring.\\

Le c\oe{}ur de Spring est une implémentation du design pattern Inversion of Control (IoC) : l'instanciation des classes n'est plus faite par le développeur mais par le framework. Permettant ainsi de l'injection de dépendance, c'est à dire, d'automatiquement injecter des objets lors de l'instanciation. Tous les modules de Spring reposent sur cette fonctionnalité. La déclaration des instances à créer par Spring se fait via un fichier XML ou via des annotations directement dans le code. On peut donc voir Spring comme une fabrique d'objets (design pattern Factory).\\

De nombreux modules font de Spring une énorme boite à outils : intégration avec un ORM (Hibernate, JPA, \dots), programmation orientée aspect (AOP), implémentation du design pattern MVC pour faciliter la création d'application web, intégration dans les frameworks de tests unitaires (JUnit), \dots

\subsection{Formation continue}

Après la phase de formation du stage terminée, j'ai ,parallèlement au travail demandé, continué et approfondi plusieurs sujets pour ma curiosité personnelle mais aussi parce que c'est un des devoirs d'un consultant du \excilysGroup{}.\\

En premier lieu, j'ai approfondi mes connaissances en Java pure pour notamment pouvoir passer l'examen d'Oracle Certified Professional Java SE 6 Programmer (OCPJP). Cet examen permet d'être certifié par Oracle comme \flqq{}maîtrisant le langage Java\frqq{} si le score obtenu est supérieur à 61 \%. Cette certification est nécessaire pour être embauché dans le \excilysGroup{}. J'ai obtenu la note de 86 \%.\\

Au cours de mon stage, j'ai aussi participé à une formation d'une journée sur Android \cite{parlezvousandroid} avec d'autres stagiaires mais aussi avec des consultants d'autres entreprises. Bien que je ne travail pas sur plateforme Android, cette formation me permet d'être capable de créer des application sur cette plateforme et donc de m'offrir plus de possibilités pour mon avenir.