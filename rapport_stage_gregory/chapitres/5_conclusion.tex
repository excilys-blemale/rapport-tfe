\chapter*{Conclusion}
\addcontentsline{toc}{chapter}{Conclusion}

Pendant ces six mois de stage ingénieur, j'ai pu vraiment apprécier le métier de consultant d'une SSII (Société de Service en Ingénierie Informatique) et ce qu'il implique. J'ai pu notamment apprendre les deux aspects les plus importants de ce métier : la formation et la communication.\\

La formation est importante dans la mesure où elle permet de rester informée des nouvelles technologies, des améliorations sur certains frameworks, \dots{} Se former, c'est aussi comprendre comment un outil fonctionne. En effet, la première impression quand on utilise un framework est de le trouver magique ! Si on s'arrête là, on ne peut pas avancer et comprendre comment bien utiliser un framework. Il est nécessaire de comprendre son fonctionnement, ses limitations et ses cas d'utilisation. Il faut donc approfondir ses connaissances car elles permettent de faire la différence sur un projet et d'apporter un vrai plus dans une équipe.\\

Je me suis ainsi rendu compte que si à la sortie de l'INSA de Rouen nous avons des connaissances sur beaucoup de concepts et le bagage nécessaire pour comprendre n'importe quel projet informatique, notre formation est insuffisante pour un véritable projet. En effet, nous n'avons aucune connaissance des frameworks à utiliser dans un projet, de leur apport et de leur nécessité.\\

D'un autre côté, la communication s'est avéré être une arme, notamment le fait de faire une réunion rapide chaque matin entre tous les stagiaires pour expliquer ce qu'on a fait la veille sur nos projets respectifs et les problèmes rencontrés. Bien qu'on ne travail pas sur le même projet, certains problèmes peuvent être indépendants ou certaines personnes peuvent les avoir déjà rencontrés. Ainsi, si une personne peut aider, cette réunion donne un coup de pouce.\\

Sur notre projet d'e-banking, cette communication s'est révélée très utile puisque simplement en parlant de notre travail, on pouvait avoir un avis extérieur et surtout des idées nouvelles.\\

Pour finir, je suis très satisfait de ce stage et de la façon dont il s'est déroulé. Il correspond totalement à mes attentes et j'en remercie le \excilysGroup{}. De plus, les personnes avec qui m'ont encadrées semblent satisfaites de mon travail puisqu'on m'a proposé de continuer avec le \excilysGroup{}.