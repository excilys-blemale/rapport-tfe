\chapter{Présentation du stage}

\section{Sujet du stage}

Le sujet du stage, tel qu'il est défini dans la convention :\\

\emph{Évolutions de systèmes d'informations Java/JEE : Le stagiaire participera au développement de notre système d'information ou à celui de l'un de nos clients dans les technologies Java/JEE. Il sera d'abord formé à l'utilisation de différents outils et frameworks, puis les mettra en \oe{}uvre dans le cadre du projet auquel il participera sous la conduite d'un architecte logiciel.}\\

Le \excilysGroup{} recrutant principalement après un stage ingénieur, les stages proposés contiennent une importante partie de formation. Cette formation permet de garantir son engagement d'excellence technique auprès de ses clients. Une première partie du stage, durant deux mois, est donc consacré à la formation. Pendant la seconde partie, les stagiaires sont amenés à participer à des projets internes.

\section{La formation}

La première partie de mon stage a donc été une période de formation durant environ deux mois. Durant cette période, les stagiaires utilisent la plateforme d'e-learning Capico, dans laquelle un grand nombre de technologies et de frameworks Java/JEE sont abordés sous forme de cours, de QCM et d'exercices pour la mise en pratique des technologies étudiées.\\

Cette partie de formation sur Capico, d'une duré d'un mois, m'a permis d'étudier le langage Java, Subversion, UML et de nombreux frameworks utilisés dans le monde de l'entreprise tels que Maven, Hibernate, Spring, Log4J, \dots{} De plus, des méthodes de gestion de projets Agiles ont été survolés telles que l'eXtreme Programming (XP).\\

Pendant le mois restant, les connaissances fraîchement apprises sont mises en pratique dans un mini-projet consistant en la réalisation d'une application web d'e-banking. Ce mini-projet est réalisé en groupe de plusieurs stagiaires sous la tutelle de Stéphane LANDELLE, directeur technique d'\ebi{}. Il intervient à la fois comme client de notre application, et comme référent technique.

\section{Capico}

Après cette phase de formation, j'ai été amené à travailler sur deux projets. Le premier étant Capico, la plateforme d'e-learning créée par \excilys{}. C'est la plateforme même qui a été utilisée pour la formation. Elle n'est pas dédiée aux stagiaires mais se veut une solution complète d'e-coaching, permettant la création de classes, d'examens, le suivi d'élèves et plus encore. Cette plateforme assez volumineuse évolue principalement avec le travail des stagiaires.\\

Mon travail sur Capico s'est découpé en trois phases que j'expliquerai plus en détail dans la suite du rapport :

\begin{itemize}
	\item Formation sur Flex,
	\item Recette de l'application et report des bogues mis en évidence,
	\item Correction des bogues.
\end{itemize}

\section{Gatling}

Le deuxième projet auquel j'ai participé pendant le reste de mon stage, soit un peu plus de 3 mois, est le projet Gatling. Gatling est une application permettant d'effectuer des tests de charge sur des applications web. Cette application est développée par Stéphane LANDELLE. Elle est open source pour la majeure partie. Elle est distribuée gratuitement et disponible pour tous sur internet. Cette application est notamment utilisée lors d'audits techniques d'applications réalisés par \ebi{}.\\

Mon travail avec deux autres stagiaires consistait à se renseigner sur les technologies à utiliser pour ajouter certaines fonctionnalités puis éventuellement les développer. Ces fonctionnalités sont :

\begin{itemize}
	\item Récupérer les métriques du serveur contenant l'application en test pour les mettre en relation avec les métriques du côté client, générée par Gatling,
	\item Exposer en temps réel les métriques mesurées par Gatling vers un système tierce de monitoring temps réel,
	\item Améliorer le système de génération de rapport pour pouvoir générer des rapports lors des long tests (plusieurs jours).\\
\end{itemize}

En parallèle, je me suis formé sur les technologies utilisées par Gatling que sont le langage de programmation multi-paradigme Scala ainsi que le framework Akka.