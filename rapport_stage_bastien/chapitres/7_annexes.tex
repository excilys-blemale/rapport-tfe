\appendix

\chapter{Charte \excilys{}}
\label{ann:charte}
\includepdf{annexes/charte_excilys.pdf}

\chapter{Test de charge de MF Banking}
\label{ann:gatling}
\begin{figure}[h!]
	\centering
		\includegraphics[scale=0.5]{images/global.pdf}
	\caption{Stastiques globales du test : nombre de requêtes réussies et échouées et détail du temps de réponse}
\end{figure}

\begin{figure}[h!]
	\centering
		\includegraphics[scale=0.5]{images/indicators.pdf}
	\caption{On peut voir que presque toutes les requêtes ont été effectuées en moins de 800 ms}
\end{figure}

\begin{figure}[h!]
	\centering
		\includegraphics[scale=0.5]{images/active-sessions.pdf}
	\caption{Nombre d'utilisateurs actifs simultanément. La montée progressive du nombre d'utilisateurs au début est dû à une \textit{rampe} : on demande à Gatling de prendre 30 secondes pour lancer progressivement les 1000 utilisateurs du test}
\end{figure}

\begin{figure}[h!]
	\centering
		\includegraphics[scale=0.5]{images/response-time.pdf}
	\caption{Distribution du temps de réponse : on peut voir que la majorité des requêtes ont été été effectuées quasi instantanément et que le nombre de requêtes lentes est tellement faible qu'elles n'apparaissent même pas sur le graphe}
\end{figure}

\begin{figure}[h!]
	\centering
		\includegraphics[scale=0.5]{images/requests.pdf}
	\caption{Nombre de requêtes qui démarrent par seconde : on observe qu'une fois la rampe terminée, le système se maintient sans difficulté entre 800 et 1200 requêtes par seconde}
\end{figure}

\begin{figure}[h!]
	\centering
		\includegraphics[scale=0.5]{images/transactions.pdf}
	\caption{Nombre de transactions (requêtes qui s'achèvent) par seconde : mêmes observations que pour les requêtes, le système se maintient durant tout le test entre 800 et 1200 transactions par seconde}
\end{figure}

\chapter{Impact des solutions de monitoring}
\label{ann:benchmark}
\begin{figure}[h!]
	\centering
		\includegraphics[scale=0.5]{images/benchmark-monitoring.pdf}
	\caption{Analyse de l'impact de différentes solutions de monitoring sur le serveur testé et sur Gatling.}
\end{figure}

\chapter{UI du Gatling Jenkins Plugin}
\label{ann:jenkins}
\begin{figure}[h!]
	\centering
		\includegraphics[scale=0.35]{images/jenkins-conf.pdf}
	\caption{Configuration du plugin dans la configuration d'un projet Jenkins. Possibilité de déclarer une liste de conditions d'échec et d'instabilité du \textit{build}.}
\end{figure}

\begin{figure}[h!]
	\centering
		\includegraphics[scale=0.25]{images/jenkins-dashboard.pdf}
	\caption{Intégration Gatling dans le dashboard d'un projet.}
\end{figure}

\begin{figure}[h!]
	\centering
		\includegraphics[scale=0.30]{images/jenkins-projet.pdf}
	\caption{Graphiques permettant de suivre l'évolution des performances du projet.}
\end{figure}

\begin{figure}[h!]
	\centering
		\includegraphics[scale=0.25]{images/jenkins-build.pdf}
	\caption{Intégration du rapport Gatling dans la page du build.}
\end{figure}