\chapter*{Conclusion}
\addcontentsline{toc}{chapter}{Conclusion}

Pendant ces six mois de stage de fin d'étude, j'ai vraiment pu appréhender le métier de consultant d'une SSII (Société de Service en Ingénierie Informatique) et ce qu'il implique. J'ai notamment pu apprendre deux des  aspects les plus importants de ce métier : la formation et la communication.\\

La formation est importante dans la mesure où elle permet de rester informé des évolutions technologiques, des améliorations sur les frameworks importants, \dots{} Se former, c'est aussi comprendre comment un outil fonctionne. En effet, aux premiers abords, un framework semble souvent \og magique \fg{} et il est donc important de comprendre son fonctionnement pour pouvoir identifier ses limitations et ses cas d'utilisation. Approfondir ses connaissances c'est donc permettre de faire la différence sur un projet et d'apporter un vrai plus dans une équipe.\\


D'un autre côté, la communication est une composante primordiale du travail de consultant, car c'est avant tout un travail d'équipe. Le bon fonctionnement d'une équipe réside principalement dans sa capacité à communiquer en interne et avec le client, c'est d'ailleurs la pierre angulaire des méthodes Agiles. Il est aussi important pour un consultant  de pouvoir exposer clairement et simplement un projet, une technologie, \dots{} Cette capacité lui permet d'être force de proposition au sein de son équipe, et être, à nouveau, un atout pour son équipe.\\

Finalement ce stage à rempli toutes les attentes que j'avais placé en lui. Il m'a réellement permis de faire la transition entre le monde de l'apprentissage et celui du travail. 