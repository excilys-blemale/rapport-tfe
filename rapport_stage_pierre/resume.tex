\section{Résumé}

\subsection{Problématiques}

Mon stage s'est axé autour de deux problématiques :
\begin{itemize}
	\item Se former aux technologies Java/JEE et développer une application web d'e-banking  en équipe afin de mettre à profit cette formation (2,5 mois)
	\item Ajouter de nouvelles fonctionnalités à Gatling, injecteur de charge open source développé par Excilys (3,5 mois)
\end{itemize}

\subsection{Intégration à l'existant}

\subsubsection*{Application d'e-banking}

Le but de l'application d'e-banking était de mettre à profit la formation initiale et de démontrer notre maîtrise des outils et technologies auxquels nous avons été formés.\\
Elle a fait l'objet d'un nouveau projet, afin de nous permettre de travailler sur chacun des composants de l'application mais également de nous offrir une certaine liberté quant aux choix techniques et d'implémentation.

\subsubsection*{Gatling}
Le projet Gatling existant depuis plusieurs mois, les apports de mon équipe ont naturellement dû s'intégrer à la base de code existante.\\
\'Etant donné que nous avons essentiellement réalisé de nouvelles fonctionnalités, l'objectif a été de limiter autant que possible de modifier le code existant, en se rapprochant d'un système de \textit{plugins}.

\subsection{Approche adoptée}

\subsubsection*{Application d'e-banking}

L'approche adoptée pour développer l'application d'e-banking a été la même pour l'ensemble des stagiaires qui devaient réaliser ce projet :
\begin{itemize}
	\item Une liste de fonctionnalités à implémenter nous a été fournie
	\item Nous devions réaliser l'ensemble des fonctionnalités en cinq semaines
	\item Nous avons travaillé en suivie la méthode de gestion de projets Scrum
	\item La qualité technique de notre travail a été vérifiée régulièrement
\end{itemize}

\subsubsection*{Gatling}

L'approche adoptée pour Gatling a été très différente.\\
Nous avons dû effectuer un important travail de recherche en amont, afin de maîtriser les nouveaux outils et technologies que nous devions utiliser et maîtriser.
Nous avions toujours une liste de fonctionnalités à implémenter, mais ce travail de recherche impliquait que nous pouvions aussi bien très vite réaliser un des fonctionnalités car les concepts à maîtriser ou les librairies à utiliser s'avéraient simples comme cela pouvait prendre plusieurs semaines pour cerner le problème et le moyen de le résoudre.

\subsection{Outils utilisés}

\subsubsection*{Application d'e-banking}

L'application d'e-banking a été développée en Java et s'est concentrée autour des trois technologies qui ont été au cœur de la formation : 
\begin{itemize}
	\item Maven, outil de build et de gestion de projet
	\item Spring, framework d'entreprise
	\item Hibernate, framework de persistance et de mapping objet-relationnel
\end{itemize}

De nombreuses autres librairies sont venues les compléter, mais l'objectif premier de ce projet restait de maîtriser ces trois technologies, omniprésentes dans les applications d'entreprise actuelles.

Nous avons également utilisé Git pour le contrôle de version et le serveur d'intégration continue Jenkins.
\subsubsection*{Gatling}

Le projet Gatling utilise de nombreuses  technologies :
\begin{itemize}
	\item Scala, le langage de programmation utilisé pour développer l'application
	\item Akka, toolkit de développement d'application concurrentes et distribuées
	\item Netty \& et Async HTTP Client, librairies permettant d'effectuer des requêtes HTTP asynchrones
	\item Metrics, toolkit facilitant l'ajout de métriques à une application et l'analyse de ces métriques via des outils de reporting
\end{itemize}
Maven a été également utilisé pour le projet Gatling.

\subsection{Apport personnel}

\subsubsection*{Application d'e-banking}

Comme les autres membres de mon équipe, j'ai pris part à l'ensemble des phases du développement de l'application d'e-banking :
\begin{itemize}
	\item Choix des fonctionnalités à implémenter à chaque sprint
	\item Choix d'architecture et des librairies à utiliser si besoin
	\item Développement et test du code produit
\end{itemize}

\subsubsection*{Gatling}

Comme pour l'application d'e-banking, j'ai pris part à l'ensemble des phases de développement du développement des nouvelles fonctionnalités de Gatling, en concertation avec le responsable du projet Gatling.

\subsection{Etat du projet à la fin du stage et perspectives}

\subsubsection*{Application d'e-banking}

L'application d'e-banking n'ayant qu'un but de formation, cette application est terminée et ne sera à priori plus modifiée.\\
Elle sert cependant toujours de référence technique, de par le nombre de technologies que nous avons mises en œuvre dans le cadre de ce projet.

\subsubsection*{Gatling}

Plusieurs des fonctionnalités souhaitées sont déjà réalisées et seront très prochainement intégrées dans une version stable de Gatling,d'autres sont encore  en cours d'écriture mais devraient être finies d'ici la fin de mon stage.\\
Je compte également poursuivre ma contribution au projet Gatling après la fin de mon stage, sur mon temps libre.